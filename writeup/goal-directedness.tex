\documentclass{article}

\usepackage[margin=0.9in]{geometry}
\usepackage{parskip}

\usepackage{mathtools,amssymb}
\usepackage{enumitem}
\usepackage{multirow}
\usepackage{booktabs}

\DeclareMathOperator*{\argmax}{\arg\max}
\DeclareMathOperator*{\Ex}{\mathbb E}
\newcommand\commentout[1]{}

\begin{document}
    \commentout{
        \begin{table}
            \centering
            \renewcommand\arraystretch{1.2}
            \begin{tabular}{r|cc}
                & \multicolumn{2}{c}{A} \\\hline
                \multirow{ 2}{*}{\rotatebox{90}{A}}&
                    %1.
                    % Specify behavior directly (the trace of a program)
                    Behavior ($\tau : S^* \to A^*$)
                    &
                    %2.
                    % Specify behavior in a higher level programming language
                    Policy ($\pi: S \to A$)
                    \\&
                    %3.
                    % Specify a metric by which you judge behavior, and train systems to maximize this metric.
                    Desirability ($S^* \times A^* \to U$)
                    &
                    % 4.
                    Reward function ($A \times S \to U$)
            \end{tabular}
        \end{table}}

    % One way of building a program is to specify its behavior directly.
    % A more compact way of doing this is to generate code, which unfolds into behavior.

    \section{Review of Markov Decision Processes}


    Let's consider an agent that, at each timestep $t$, observes the state $s^{(t)} \in S$ of the system, and takes an action $a^{(t)} \in A$.
    Suppose further that the system dynamics are controlled by a fixed \emph{transition} map $\tau: S \times A \to \Delta S$ such that, for all $t$, the next state $s^{(t+1)}$ is drawn independently from the distribution $\tau(s^{(t)}, a^{(t)})$.

    % How should one go about specifying the behavior of an agent in this system?
    The agent's behavior is the sequence of contextual actions
    % $(s^{(t)}, a^{(t)})_{t = 1, 2, \ldots}$.
    $\{ s^{(t)} \mapsto a^{(t)} \}_{t=1,2,\ldots}$.
    How should one go about specifying an agent's behavior in this setting?
    One option is to specify each action directly; another is to apply a policy $\pi: S \to \Delta A$ that gives a distribution of actions for each state.


    The more common approach is to instead specify a \emph{reward function} $R : S \times A \times S \to \mathbb R$, and try to automatically learn the policy which maximizes total reward, $\sum_{t} R(s^{(t)}, a^{(t)}, s^{(t+1)})$.
    Because it is hard to reason about this sum when it diverges, it is typically more convenient to choose a \emph{discount} rate $\gamma \in (0,1)$ , and find a policy that maximizes maximize \emph{discounted} reward:
    $
        \sum_{t} \gamma^t R(s^{(t)}, a^{(t)}, s^{(t+1)})
    $.

    This motivates the standard definition of a Markov Decision Process (MDP) $(S, A, \gamma, \tau, R)$, and the notion of ``solving'' an MDP, which is to say, finding an optimal policy $\pi^* : S \to \Delta A$ which maximizes the total discounted reward.
    % \[
    %     \argmax_{\pi : S \to \Delta A}
    %         \sum_{t=1}^{\infty}\Ex_{a \sim \pi} R()
    % \]

    Given a fixed policy $\pi$ and reward function $R$, we can define a \emph{Value} function $V_{\pi, R} : S \to \mathbb R$ its expected future reward.
    It satisfies the recursive relation
    \begin{align*}
        V_{\pi,R}(s) = \Ex_{ \substack{ a \sim \pi(A|s) \\ s' \sim \tau(S|a,s) }}
            \Big[ R(s,a,s') + \gamma V_{\pi, R}(s') \Big]
    \end{align*}
    There is also a standard notion of an action-dependent value, or $Q$-value,
    \[
        Q(s,a) = \Ex_{ s' \sim \tau(S|a,s) } \Big[ R(s,a,s') + \gamma V_{\pi, R}(s') \Big]
    \]
    that does not have the dependence on .
    It satisfies the recursive relation
    \begin{align*}
        V_{\pi,R}(s) = \Ex_{ \substack{ a \sim \pi(A|s) \\ s' \sim \tau(S|a,s) }}
            \Big[ R(s,a,s') + \gamma V_{\pi, R}(s') \Big]
        Q(a,s) = \Ex_{} \Ex_{s' \sim \tau(S|a,s)} \Big[ R(s,a,s') + \gamma V^*_R(s') \Big]
    \end{align*}

    There's also a notion of value
    \[
        V^*_R(s)  = \max_{a \in A}
    \]




    \section{Notions of Goal-Directedness}



    \section{Experiments}




\end{document}
