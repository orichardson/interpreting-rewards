\documentclass{article}

\usepackage[margin=0.9in]{geometry}
\usepackage{mathtools,amssymb}
\usepackage{enumitem}
\usepackage{multirow}
\usepackage{booktabs}

\newcommand\commentout[1]{}
\begin{document}  
    
    \commentout{
        \begin{table}
            \centering
            \renewcommand\arraystretch{1.2}
            \begin{tabular}{r|cc}
                & \multicolumn{2}{c}{A} \\\hline
                \multirow{ 2}{*}{\rotatebox{90}{A}}&
                    %1.
                    % Specify behavior directly (the trace of a program) 
                    Behavior ($\tau : S^* \to A^*$)
                    & 
                    %2.
                    % Specify behavior in a higher level programming language
                    Policy ($\pi: S \to A$)
                    \\&
                    %3.
                    % Specify a metric by which you judge behavior, and train systems to maximize this metric.
                    Desirability ($S^* \times A^* \to U$)
                    &
                    % 4. 
                    Reward function ($A \times S \to U$)
            \end{tabular}
        \end{table}}
    
    % One way of building a program is to specify its behavior directly. 
    % A more compact way of doing this is to generate code, which unfolds into behavior.
    
    \section{Review of Markov Decision Processes}
    

    Let's consider an agent that, at each timestep $t$, observes the state $s^{(t)} \in S$ of the system, and takes an action $a^{(t)} \in A$. 
    Suppose further that the system dynamics are controlled by a fixed \emph{transition} map $\tau: S \times A \to \Delta S$ such that, for all $t$, the next state $s^{(t+1)}$ is drawn independently from the distribution $\tau(s^{(t)}, a^{(t)})$.
    % How should one go about specifying the behavior of an agent in this system?  
    The agent's behavior is the sequence of actions $(s^{(t)}, a^{(t)})_{t = 1, 2, \ldots}$. 
    
    This motivates the standard definition of a Markov Decision Process (MDP) $(S, A, \gamma, \tau, R)$. 
        
    
    
    \section{Notions of Goal-Directedness}
    
    Suppose you have a 
    
    
    
    
    
\end{document}
